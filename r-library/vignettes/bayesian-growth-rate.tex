% Options for packages loaded elsewhere
\PassOptionsToPackage{unicode}{hyperref}
\PassOptionsToPackage{hyphens}{url}
%
\documentclass[
]{article}
\usepackage{lmodern}
\usepackage{amssymb,amsmath}
\usepackage{ifxetex,ifluatex}
\ifnum 0\ifxetex 1\fi\ifluatex 1\fi=0 % if pdftex
  \usepackage[T1]{fontenc}
  \usepackage[utf8]{inputenc}
  \usepackage{textcomp} % provide euro and other symbols
\else % if luatex or xetex
  \usepackage{unicode-math}
  \defaultfontfeatures{Scale=MatchLowercase}
  \defaultfontfeatures[\rmfamily]{Ligatures=TeX,Scale=1}
\fi
% Use upquote if available, for straight quotes in verbatim environments
\IfFileExists{upquote.sty}{\usepackage{upquote}}{}
\IfFileExists{microtype.sty}{% use microtype if available
  \usepackage[]{microtype}
  \UseMicrotypeSet[protrusion]{basicmath} % disable protrusion for tt fonts
}{}
\makeatletter
\@ifundefined{KOMAClassName}{% if non-KOMA class
  \IfFileExists{parskip.sty}{%
    \usepackage{parskip}
  }{% else
    \setlength{\parindent}{0pt}
    \setlength{\parskip}{6pt plus 2pt minus 1pt}}
}{% if KOMA class
  \KOMAoptions{parskip=half}}
\makeatother
\usepackage{xcolor}
\IfFileExists{xurl.sty}{\usepackage{xurl}}{} % add URL line breaks if available
\IfFileExists{bookmark.sty}{\usepackage{bookmark}}{\usepackage{hyperref}}
\hypersetup{
  pdftitle={Bayesian growth rate estimation},
  pdfauthor={Rob Challen},
  hidelinks,
  pdfcreator={LaTeX via pandoc}}
\urlstyle{same} % disable monospaced font for URLs
\usepackage[margin=1in]{geometry}
\usepackage{graphicx,grffile}
\makeatletter
\def\maxwidth{\ifdim\Gin@nat@width>\linewidth\linewidth\else\Gin@nat@width\fi}
\def\maxheight{\ifdim\Gin@nat@height>\textheight\textheight\else\Gin@nat@height\fi}
\makeatother
% Scale images if necessary, so that they will not overflow the page
% margins by default, and it is still possible to overwrite the defaults
% using explicit options in \includegraphics[width, height, ...]{}
\setkeys{Gin}{width=\maxwidth,height=\maxheight,keepaspectratio}
% Set default figure placement to htbp
\makeatletter
\def\fps@figure{htbp}
\makeatother
\setlength{\emergencystretch}{3em} % prevent overfull lines
\providecommand{\tightlist}{%
  \setlength{\itemsep}{0pt}\setlength{\parskip}{0pt}}
\setcounter{secnumdepth}{-\maxdimen} % remove section numbering

\title{Bayesian growth rate estimation}
\author{Rob Challen}
\date{22/10/2021}

\begin{document}
\maketitle

\hypertarget{bayesian-growth-rate-estimation}{%
\section{Bayesian growth rate
estimation}\label{bayesian-growth-rate-estimation}}

Assume \(I_0, I_1, \dots, I_t\) is a time series of infection counts,
assumed to be drawn from some discrete probability distribution with
expected value \(\overline{I_t}\). Assume, as before \(I_t\) is a
Poisson distributed quantity, with a rate parameter which is a function
of time:

\[
\begin{aligned}
E[I_t] &= \lambda_t \\
\end{aligned}
\]

Assuming \(\lambda_t\) is constant over \([t-\tau;t+\tau]\) then by
definition:

\[
\begin{aligned}
P(I_{t-\tau},\dots,I_{t+\tau}|\lambda_t) = \prod_{s=t-\tau}^{t+\tau}\frac{e^{-\lambda_t}\lambda_t^{I_s}}{I_s!} \\
\end{aligned}
\]

If we assume \(\lambda_t\) to be gamma distributed with shape parameter
\(\alpha\) and rate parameter \(\beta\), and if \(n = 2\tau+1\)

\[
\begin{aligned}
P(I_{t-\tau},\dots,I_{t+\tau}|\lambda_t) &= \frac{e^{-n\lambda_t}\lambda_t^{\sum{I_s}}}{\prod_{s=t-\tau}^{t+\tau}I_s!} \\
P(\lambda_{t}) &= \frac{\beta^\alpha}{\Gamma(\alpha)} \lambda_{t}^{\alpha-1}e^{-\beta\lambda_{t}} \\
P(\lambda_t|I_{t-\tau},\dots,I_{t+\tau}) &= \frac{P(I_{t-\tau},\dots,I_{t+\tau}|\lambda_t)P(\lambda_{t})}{P(I_{t-\tau},\dots,I_{t+\tau})}\\
P(\lambda_t|I_{t-\tau},\dots,I_{t+\tau}) &= \frac{e^{-n\lambda_t}\lambda_t^{\sum{I_s}}}{\prod_{s=t-\tau}^{t+\tau}I_s!}\frac{\beta^\alpha}{\Gamma(\alpha)} \lambda_{t}^{\alpha-1}e^{-\beta\lambda_{t}}\\
P(\lambda_t|I_{t-\tau},\dots,I_{t+\tau}) &\propto \lambda_t^{\sum{I_s+\alpha-1}}e^{-(2\tau+1+\beta)\lambda_t} \\
P(\lambda_t|I_{t-\tau},\dots,I_{t+\tau}) &\sim \Gamma\big(\sum_{t-\tau}^{t+\tau}{I_s}+\alpha, 2\tau+1+\beta\big)\\
\alpha' &= \alpha+\sum_{t-\tau}^{t+\tau}{I_s}\\
\beta' &= 2\tau+1+\beta
\end{aligned}
\]

The posterior estimate of the Poisson rate \(\lambda\) is gamma
distributed by definition but to estimate the likely values of \(I_t\)
(\(\overline{I_t}\)) we need the posterior predictive distribution:

\[
\begin{aligned}
\overline{I_t} &\sim NegBin\Big(\alpha',\frac{\beta'}{\beta'+1}\Big)\\
E(\overline{I_t}|I_{t-\tau},\dots,I_{t+\tau}) &= \frac{\alpha'}{\beta'} \\
V(\overline{I_t}|I_{t-\tau},\dots,I_{t+\tau}) &= \alpha'\Big(\frac{\beta'+1}{\beta'^2}\Big)
\end{aligned}
\]

There is probably something interesting we can do with the
\(P(\overline{I_t}|I_{t-\tau} \dots I_{t+\tau})\) to detect importation
or anomaly events

\hypertarget{growth-rate}{%
\section{Growth rate}\label{growth-rate}}

The exponential growth rate \(r_t\) is the gradient of the logarithm of
I (\(\frac{d}{dt}log(\overline{I_t})\))

\[
\begin{aligned}
r_t \approx \frac{1}{2m}(log(E(I_{t+m}))-log(E(I_{t-m}))) \\
r_t = \frac{1}{2m}log\frac{\lambda_{t+m}}{\lambda_{t-m}}
\end{aligned}
\] if \(m = \tau\) and \(\phi_t = e^{2\tau r_t}\)

\$\$ \textbackslash begin\{aligned\}

y = g(r\_t) \&= e\^{}\{2\tau r\_t\}\textbackslash{} r\_t =
g\^{}\{-1\}(y) \&= \frac{1}{2\tau}log(y)\textbackslash{}
\textbackslash end\{aligned\} \$\$

\[
\begin{aligned}
r_t &= \frac{1}{2\tau}log\frac{\lambda_{t+\tau}}{\lambda_{t-\tau}} \\
g(r_t) &\sim \Bigg(\frac{Gamma(\alpha+\sum_{s=t}^{t+2\tau}{I_s}, \beta')}{Gamma(\alpha+\sum_{r=t-2\tau}^{t}{I_r}, \beta')}\Bigg) \\
\alpha'_{\tau+} &= \alpha+\sum_{t}^{t+2\tau}{I_s}\\
\alpha'_{\tau-} &= \alpha+\sum_{t-2\tau}^{t}{I_s}\\
g(r_t) &\sim \Bigg(\frac{
Gamma(\alpha'_{\tau+}, \beta')
}{
Gamma(\alpha'_{\tau-}, \beta')
}
\Bigg)\\
\end{aligned}
\]

Ratio of 2 gammas with same rate parameter is a BetaPrime
(\url{https://en.wikipedia.org/wiki/Beta_prime_distribution})

\$\$ \textbackslash begin\{aligned\} r\_t \sim Y \&=
\frac{1}{2\tau}log\Big(BetaPrime\big(\alpha'\emph{\{\tau+\},
\alpha'}\{\tau-\})\Big) \textbackslash{} g(r\_t) \sim X \&=
BetaPrime\big(\sum\emph{\{t\}\^{}\{t+2\tau\}\{I\}+\alpha,
\sum}\{t-2\tau\}\^{}\{t\}\{I\}+\alpha\big) \textbackslash{}

BetaPrime(\alpha\_1, \alpha\_2) \&: \textbackslash{} f(x) \&=
\frac{1}{B(\alpha_1,\alpha_2)}x\^{}\{\alpha\_1-1\}(1+x)\^{}\{-\alpha\_1-\alpha\emph{2\}
\textbackslash{} F(x) \&= I}\{ \frac{x}{1+x}\}(\alpha\_1,\alpha\_2)
\textbackslash end\{aligned\} \$\$ Where \(I\) is the regularised beta
function and \(B\) is the complete beta function.

support for \(x\) here is 0,inf as we are looking at ratio of gammas.
Support for \(r_t\) is -inf,inf.

given that \(g(r_t)\) can be differentiated and is a strictly increasing
function we note:

Give that Strictly increasing functions / Method of transformations:
\url{https://www.statlect.com/fundamentals-of-probability/functions-of-random-variables-and-their-distribution}
\url{https://www.probabilitycourse.com/chapter4/4_1_3_functions_continuous_var.php}

\$\$ Y = \frac{1}{2\tau}log(X) \textbackslash{} g(x) =
\frac{1}{2\tau}log(x) \textbackslash{} g\^{}\{-1\}(y) = e\^{}\{2\tau y\}
\textbackslash{} \frac{dg^{-1}}{dy} =
2\tau e\^{}\{2\tau y\}\textbackslash{}

F\_Y(y) = F\_X(g\^{}\{-1\}(y)) \textbackslash{} f\_Y(y) =
f\_X(g\^{}\{-1\}(y)) \frac{dg^{-1}(y)}{dy}

\$\$

Giving a pdf for the posterior estimate of \(r_t\): \[
\begin{aligned}
f_Y(r_t) &= f_X(g^{-1}(r_t))\frac{dg^{-1}(r_t)}{dr_t}\\
f_Y(r_t) &= 2\tau e^{2\tau r_t}f_X(e^{2\tau r_t})\\
f_Y(r_t) &=  2\tau e^{2\tau r_t}\frac{
\Big(
  e^{r_t(\alpha'_{\tau+}-1)}
  (1+e^{r_t})^{(-\alpha'_{\tau+}-\alpha'_{\tau-})}
\Big)}{B(\alpha'_{\tau+}, \alpha'_{\tau-})}\\
f_Y(r_t) &=  \frac{2\tau}{B(\alpha'_{\tau+}, \alpha'_{\tau-})}
\Big(
  e^{r_t(\alpha'_{\tau+}+2\tau-1)}
  (1+e^{r_t})^{(-\alpha'_{\tau+}-\alpha'_{\tau-})}
\Big)\\
\end{aligned}
\]

And a cdf for the posterior estimate of \(r_t\):

\[
\begin{aligned}
F_Y(r_t) &= F_Y(g^{-1}(r_t))\\
F_X(r_t) &= F_Y(e^{2\tau r_t})\\
F_X(r_t) &= I_{
\frac{e^{2\tau r_t}}{1+e^{2\tau r_t}}}(\alpha_{\tau+},\alpha_{\tau-})
\end{aligned}
\]

where \(I_x(a,b)\) is the regularised beta function.

\url{https://mathworld.wolfram.com/RegularizedBetaFunction.html}
\url{https://commons.apache.org/proper/commons-math/javadocs/api-3.3/org/apache/commons/math3/special/Beta.html}
\url{https://en.wikipedia.org/wiki/Beta_prime_distribution\#Properties}

I think (but I am not sure) that the quantile function for \(g(r_t)\)
can simply be transformed backwards to give a quantile estimate for
\(r_t\). This makes intuitive sense to me and looks OK experimentally.
This would also suggest that we can sample from \(Y\) and transform
using \(g^{-1}(y)\) to get unbiased samples of \(r_t\).

\hypertarget{r_t-from-poisson-rate}{%
\section{R\_t from Poisson rate}\label{r_t-from-poisson-rate}}

If \(I_t \sim Poisson(\lamda_t)\) and an estimate of \(\lambda_t\) is
available. \(\omega_1, \omega_2, \dots, \omega_s\) is another
probability distribution, the infectivity profile, that defines the
likelihood that a case infected at time \(t\) resulted from a case
infected before time \(t-s\). This definition implies that
\(\omega_0 = 0\), and that discrete time measures represent the upper
bound of the equivalent continuous unit time interval, rather than, for
example, the middle of the interval.

\$\$ \overline{I_t} = \lambda\_t\textbackslash{} R\_t =
\frac{\text{secondary cases}}{\text{contributing primary cases}}
\textbackslash{}

R\_t = \frac{\lambda_t}{\sum_{s=1}^t \lambda_{t-s}\omega_s} \$\$

since the denominator is all poisson distributions we can combine to
give@

\[
R_t \sim \frac{Poisson(\lambda_t)}{Poisson(\sum_{s=1}^t \lambda_{t-s}\omega_s)}
\]

N.b. There is a finite probability of extinction in which case the
denominator is zero:

\$\$

P(\text{extinction}\emph{t) = P(\text{contributing primary cases} =
0)\textbackslash{} P(\text{extinction}\emph{t) =
P(Poisson(\sum}\{s=1\}\^{}t \lambda}\{t-s\}\omega\emph{s) =
0)\textbackslash{} P(\text{extinction}\emph{t) = \Big(\sum}\{s=1\}\^{}t
\lambda}\{t-s\}\omega\emph{s\Big)e\^{}\{-\Big(\sum}\{s=1\}\^{}t
\lambda\_\{t-s\}\omega\_s\Big)\} \$\$

Although PDF is not defined in the general case Cumulative probability
is defined:

\url{https://stats.stackexchange.com/questions/10951/what-is-the-distribution-of-the-ratio-of-two-poisson-random-variables}

\[
\mathbb{P}\left[\frac{X}{Y} \leq r \right] := \mathbb{P}\left[X \leq r Y\right]\\
= \sum_{y = 0}^\infty \sum_{x=0}^{\left\lfloor ry \right\rfloor} \frac{\lambda_{2}^y }{y!}e^{-\lambda_2} \frac{\lambda_{1}^x }{x!}e^{-\lambda_1}
\] And the PDF : ``The density follows from the Radon-Nykodym theorem.''
\url{https://en.wikipedia.org/wiki/Radon\%E2\%80\%93Nikodym_theorem} But
this is mind-boggling

There is a bayesian estimator of this ratio: ( I think it is the
``simple ratio'' bit of this paper:)
\url{https://arxiv.org/pdf/astro-ph/0606247.pdf} but it maybe needs to
be reworked here. There is no closed form solution to it.

The wikipedia entry has forms for mean and variance of this ratio:
\url{https://en.wikipedia.org/wiki/Ratio_distribution\#Poisson_and_truncated_Poisson_distributions}

\hypertarget{r_t-from-gamma-posterior-estimate-of-poisson-rate-this-maybe-wrong}{%
\section{R\_t from Gamma posterior estimate of poisson rate (This maybe
wrong)}\label{r_t-from-gamma-posterior-estimate-of-poisson-rate-this-maybe-wrong}}

Damn it this is maybe wrong, because the posterior distribution of
\(\lambda_t\) is not equivalent to \(\overline{I_t}\) distribution. If
we were going to use this we would need to use the posterior predictive
distribution (NegBinomial), or maybe we can do this as an approximation:

If \(I_t \sim Poisson(\lamda_t)\) and an estimate of \(\lambda_t\) is
available.

\[
R_t = \frac{\lambda_t}{\sum_{s=1}^t \lambda_{t-s}\omega_s}
\]

and the posterior distribution of
\(\lambda_t \sim Gamma(\alpha',\beta')\) as described above. Considering
the denominator as the sum of scaled gamma distributions we can say the
distribution of the denominator is a mixture distribution of Gammas:

\[
\lambda_{t-s}\omega_s \sim Gamma\big(\alpha'_{t-s}, \frac{\beta'_{t-s}}{\omega_s}\big)\\ 
\] We can use the Welch-Satterwaite equation to generate an
approximation for this mixture as another gamma distribution. In the
case of a set of gamma distributions, with parameters \(\alpha_i\) and
\(\beta_i\), an approximation of the mixture is another gamma
distribution with parameters \(\alpha_{sum}\) and \(\beta_{sum}\):
\url{https://en.wikipedia.org/wiki/Welch\%E2\%80\%93Satterthwaite_equation}
\url{https://stats.stackexchange.com/questions/72479/generic-sum-of-gamma-random-variables}

\$\$ \alpha\_\{sum\} = \frac{
\Big(\sum_i \frac{\alpha_i}{\beta_i}\Big)^2
}{
\sum_i \frac{\alpha_i}{\beta_i^2}
}\textbackslash{}

\beta\_\{sum\} = \frac{\alpha_{sum}}{
\sum_i \frac{\alpha_i}{\beta_i}
} \$\$ Using this approximation we estimate \(R_t\) to be distributed as
the ratio of 2 Gamma distributions where:

\$\$ R\_t \sim \frac{
  \frac{1}{\beta_t'}Gamma(\alpha_t',1)
}{
  \frac{1}{\beta_t''}Gamma(\alpha_t'',1)
}\textbackslash{}

R\_t
\sim BetaPrime(\alpha\_t',\alpha\_t'',1,\frac{\beta_t''}{\beta_t'})\textbackslash{}

\text{Where given: } s \in (1 .. t) \textbackslash{}

\alpha''\_t = \frac{
  \Big(\sum_s \frac{\alpha'_{t-s}\omega_s}{\beta'_{t-s}}\Big)^2
}{
  \sum_s \frac{\alpha'_{t-s}\omega_s^2}{(\beta'_{t-s})^2}
}\textbackslash{}

\beta''\_t = \frac{
  \alpha''_t
}{
  \sum_s{\frac{\alpha'_{t-s}\omega_s}{\beta'_{t-s}}}
} \$\$ This form of \(R_t\) assumes the gamma posterior estimate and so
can be applied directly to the posteriors from the earlier part of this
method. This could overestimate confidence if the denominator of the
ratio is less variable that the mixture in reality. This could be the
case and the Welch-Satterbach estimator works best when the
distributions are not completely different. However in this case the
scaling of the \(\lambda_t\) estimates by the infectivity profile does
make it quite likely the mixture of distributions will have dissimilar
means and variances. (note to self when implementing - this is weighted
sum not weighted average as done previously)

\end{document}
